% Define the subtitle of the page
\title{Unsupervised learning}

% Begin the content of the page
\subsection{Unsupervised learning}

In unsupervised learning, the task is to infer hidden structure from
unlabeled data, comprised of training examples $\{x_n\}$.

We demonstrate how to do this in Edward with an example.
The script is available
\href{https://github.com/blei-lab/edward/blob/master/examples/mixture_gaussian.py}
{here}.


\subsubsection{Data}

Use a simulated (pre-generated) dataset of datapoints
$\mathbf{x}_n\in\mathbb{R}^2$.
\begin{lstlisting}[language=Python]
x = np.loadtxt('data/mixture_data.txt', dtype='float32', delimiter=',')
data = {'x': x}
\end{lstlisting}


\subsubsection{Model}

Posit the model as a mixture of Gaussians. For more details on the
model, see the
\href{tut_mixture_gaussian.html}
{Mixture of Gaussians tutorial}.

Here we build the model in Edward using TensorFlow, and set the number
of mixture components to be 2.
\begin{lstlisting}[language=Python]
class MixtureGaussian:
    """
    Mixture of Gaussians

    p(x, z) = [ prod_{n=1}^N N(x_n; mu_{c_n}, sigma_{c_n}) Multinomial(c_n; pi) ]
              [ prod_{k=1}^K N(mu_k; 0, cI) Inv-Gamma(sigma_k; a, b) ]
              Dirichlet(pi; alpha)

    where z = {pi, mu, sigma} and for known hyperparameters a, b, c, alpha.

    Parameters
    ----------
    K : int
        Number of mixture components.
    D : float, optional
        Dimension of the Gaussians.
    """
    def __init__(self, K, D):
        self.K = K
        self.D = D
        self.n_vars = (2*D + 1) * K

        self.a = 1
        self.b = 1
        self.c = 10
        self.alpha = tf.ones([K])

    def log_prob(self, xs, zs):
        """Return a vector [log p(xs, zs[1,:]), ..., log p(xs, zs[S,:])]."""
        N = get_dims(xs['x'])[0]
        pi, mus, sigmas = zs
        log_prior = dirichlet.logpdf(pi, self.alpha)
        log_prior += tf.reduce_sum(norm.logpdf(mus, 0, np.sqrt(self.c)), 1)
        log_prior += tf.reduce_sum(invgamma.logpdf(sigmas, self.a, self.b), 1)

        # Loop over each sample zs[b,:]
        log_lik = []
        n_samples = get_dims(zs[0])[0]
        for s in range(n_samples):
            log_lik_z = N*tf.reduce_sum(tf.log(pi), 1)
            for k in range(self.K):
                log_lik_z += tf.reduce_sum(multivariate_normal.logpdf(xs['x'],
                    mus[s, (k*self.D):((k+1)*self.D)],
                    sigmas[s, (k*self.D):((k+1)*self.D)]))

            log_lik += [log_lik_z]

        return log_prior + tf.pack(log_lik)

model = MixtureGaussian(K=2, D=2)
\end{lstlisting}


\subsubsection{Inference}
Perform variational inference.
%
The latent variables are the mixture probabilities,
component means, and component variances.
Define the variational model to be a Dirichlet $\times$ fully factorized normal
$\times$ fully factorized inverse Gamma.
%The latent variables are $\mathbf{z} = (\pi, \mu, \sigma)$.
%\begin{align*}
%  q(\mathbf{z} \;;\; \lambda)
%  &=
%  \text{Dirichlet}(\mathbf{z}_\pi)
%  \times
%  \mathcal{N}(\mathbf{z}_\mu)
%  \times
%  \text{InvGamma}(\mathbf{z}_\sigma)
%\end{align*}
%
%The model in Edward is
\begin{lstlisting}[language=Python]
variational = Variational()
variational.add(Dirichlet(model.K))
variational.add(Normal(model.K*model.D))
variational.add(InvGamma(model.K*model.D))
\end{lstlisting}

Run mean-field variational inference for 500 iterations, using a batch
of 5 datapoints and 5 latent variable samples per iteration.
\begin{lstlisting}[language=Python]
inference = ed.MFVI(model, variational, data)
inference.run(n_iter=500, n_samples=5, n_minibatch=5)
\end{lstlisting}
In this case
\texttt{MFVI} defaults to minimizing the
$\text{KL}(q\|p)$ divergence measure using the score function
gradient.
For more details on inference, see the \href{tut_KLqp.html}{$\text{KL}(q\|p)$ tutorial}.


%\subsubsection{Criticism}
