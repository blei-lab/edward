% Define the subtitle of the page
\title{Probability models}

% Begin the content of the page
\subsection{Probability models}

This tutorial introduces the class of models supported in Edward.

%\subsubsection{Probability models}

A probability model has two components. The \emph{likelihood}
\begin{align*}
  p(x \mid z)
\end{align*}
is a probability distribution that describes how the data $x$ depend on some
latent random variables $z$. The likelihood models a random data generating
process, where the data $x$ are observed random variables conditioned on a
particular hidden pattern described by $z$. For example, consider measuring your
weight on a cheap scale; the likelihood model would describe the imprecisions
of the scale that reflect into the measurements.

The \emph{prior}
\begin{align*}
  p(z)
\end{align*}
is another probability distribution that captures our prior beliefs on
the latent random variables $z$. It describes the kind of hidden patterns we
expect to have produced the data $x$. Continuing our example, the prior would
capture your prior belief about your (latent) weight. A potential prior might be
a normal distribution centered at the average weight of adults on Earth (
which is $\approx62$ kilograms).

Combining the likelihood and the prior specifies a probability model,
\begin{align*}
  p(x,z)
  &=
  p(x \mid z)
  p(z)
\end{align*}
as a joint density over observed and latent random variables.

For details on how to specify a model in Edward, see the
\href{api/models_models.html}{model API}. We describe several examples in detail
in the
other model \href{tutorials.html}{tutorials}.
