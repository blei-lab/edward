% Define the subtitle of the page
\title{Developer Process}

% Begin the content of the page
\subsection{Developer Process}

\subsubsection{Standards}

\begin{itemize}
\item
  \textbf{Pull request and branches.}
  Contribute by developing on a clone of the repo and writing the code
  in a branch. Submit a pull request when ready.

  For developers with push permission to the repo, see
  \href{https://github.com/stan-dev/stan/wiki/Developer-Process#information-to-include-in-pull-request}{Stan's
  process} for how to name branches.  Do not merge your own pull
  requests or ever push to master. Someone should always review your
  code.  After merging (or deciding to close the request without
  merging), always delete the branch from the repo.
\item
  \textbf{Unit testing.} Unit testing is awesome. It's useful not only
  for checking code after having written it, but also in checking code
  as you are developing it. If you're informally writing short scripts
  that output various things anyways, I suggest saving the file in the
  \texttt{tests/} directory. This gets the momentum going as the test
  becomes formalized. For testing, simply run
  \href{http://nose2.readthedocs.io/en/latest/getting_started.html}{\texttt{nose2}}
  in the repo; it will automatically find and run all tests in
  \texttt{tests/}. Most if not all pull requests should have unit tests.
\item
  \textbf{Issue labeling system.} We use
  \href{https://github.com/stan-dev/stan/issues}{Stan's labeling system}.
  While several labels obviously don't apply to us, it's better than the
  default labels.
\end{itemize}

\subsubsection{Coding}

\begin{itemize}
\item
  \textbf{Style guidelines.}
  Follow
  \href{https://www.tensorflow.org/versions/r0.9/how_tos/style_guide.html}{TensorFlow},
  including its
  \href{https://www.tensorflow.org/versions/r0.9/how_tos/documentation/index.html}{documentation guidelines}.
  The only exceptions are detailed below. Some below are not
  necessarily exceptions but simply emphasize style guidelines you
  should be following from TensorFlow or PEP 8 anyways.
\item
  Use four-space indents rather than two-space.
\item
  To organize imports in a script, use three blocks: 1. any \texttt{from __future__
  import [...]} lines; 2. any \texttt{import [...]} lines; and 3. any
  \texttt{from [...] import [...]} lines. Each block is separated by a blank line,
  and within each block the lines are sorted alphabetically.
\item
  \texttt{edward.stats} uses SciPy standards. This
  includes, for example, the argument specification and the choice of
  how a distribution is parameterized. \texttt{edward.models}
  uses \texttt{tf.contrib.distributions} standards.
\item
  For arguments that are positive integers, use \texttt{n\_}, e.g.,
  \texttt{n\_minibatch}, \texttt{n\_print}, to represent ``number of
  {[}\ldots{}{]}''.  For class attributes that are booleans, use
  \texttt{is_}, e.g., \texttt{is_reparameterized},
  \texttt{is_multivariate}.
\item
  Aim for 70 characters per line, with some exceptions.
\item
  \href{http://programmers.stackexchange.com/questions/75919/should-package-names-be-singular-or-plural}{Package
  names are almost always plural, with the exception of} \texttt{util.py}.
\item
  Use a blank line to separate the end of an indented procedure:
\begin{lstlisting}[language=Python]
for i in range(5):
    do_stuff()

more_code()
\end{lstlisting}
\end{itemize}


\subsubsection{Suggested workflow}\label{suggested-workflow}

\begin{itemize}
\item
  \textbf{Update your Python path.} As you make changes, it can be
  cumbersome to constantly reinstall the package to test these changes.
  We recommend adding the path to your local repo to your
  \texttt{PYTHONPATH}. That is, run the following on the command-line:

\begin{lstlisting}[class=JSON]
export PYTHONPATH="${PYTHONPATH}:/path/to/repo"
\end{lstlisting}

  For this to work permanently, add this line to your
  \texttt{\textasciitilde{}/.bashrc} if you use Bash or
  \texttt{\textasciitilde{}/.zshenv} if you use zsh. Any time you import
  the package, it will look for the package locally via this directory
  path.
\item
  \textbf{Local installation.} Sometimes it does make sense to check the
  installation. To install locally, run the following when at the parent
  directory of the repo:

\begin{verbatim}
pip install -e edward
\end{verbatim}

  (We recommend not installing with \texttt{sudo}; rather
  \href{http://docs.python-guide.org/en/latest/starting/install/osx/}{use
  virtualenv}.)
\item
  \textbf{Packaging and submitting to PyPI.} First, update the version
  number in \texttt{setup.py}. Second, follow
  \href{https://packaging.python.org/en/latest/distributing/\#packaging-your-project}{these
  steps}. For shorthand, the sequence of commands is

\begin{lstlisting}[class=JSON]
python setup.py sdist
python setup.py bdist_wheel
twine upload dist/*
\end{lstlisting}

  Third, tag the release on Github and note the new additions when
  tagging this release. You can do this by comparing commits from the
  previous tagged release to master. A link that compares tagged commits
  to master is available on the
  \href{https://github.com/blei-lab/edward/releases}{releases page}.
\item
  In general, all remaining tasks to be completed are raised as Github
  issues. In other cases, write a TODO comment for things that need to
  be done but are so minor you'd rather not raise a Github issue; this
  can be helpful as you are writing code in a branch. None of the
  intermediate commits which have a TODO comment should still have
  that comment when submitting the pull request.
\end{itemize}

\subsubsection{Suggested private
workflow}\label{suggested-private-workflow}

To develop work on a branch privately, we suggest using a private repo
that maintains the master branch from the public repo. Development
happens on the private repo's branch, and when it is finished, you can
push it to the public repo's branch to submit as a pull request. We
describe this in detail.

Clone the private repo so you can work on it (create a repo if it does
not exist).

\begin{lstlisting}[class=JSON]
git clone https://github.com/blei-lab/edward-private.git
\end{lstlisting}

Pull changes from the public repo. This will let the private repo have
the latest code from the public repo on its master branch.

\begin{lstlisting}[class=JSON]
cd edward-private
git remote add public https://github.com/blei-lab/edward.git
git pull public master # Creates a merge commit
git push origin master
\end{lstlisting}

Now create your branch on the private repo, develop stuff, and pull any
latest changes from the public repo as you develop
(\texttt{git\ pull\ public\ master}). Make sure that as you're running
Edward, you're using the Edward library pointing to the private repo so
it reflects your developer changes and not pointed to the public repo
where it won't see any changes.

Finally, to create a pull request from a private repo's branch to the
public repo, push the private branch to the public repo.

\begin{lstlisting}[class=JSON]
git clone https://github.com/blei-lab/edward.git
cd edward
git remote add private https://github.com/blei-lab/edward-private.git
git checkout -b pull_request_yourname
git pull private master
git push origin pull_request_yourname
\end{lstlisting}

Now simply create a pull request via the Github UI on the public repo.
Once project owners review the pull request, they can merge it.
\href{http://stackoverflow.com/questions/10065526/github-how-to-make-a-fork-of-public-repository-private/30352360\#30352360}{Source}
