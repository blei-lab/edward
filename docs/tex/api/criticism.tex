\title{Criticism}

{{navbar}}

\subsubsection{Criticism}

Criticizing models and their inference is a crucial step in analysis.
Following falsificationists such as Popper and Box, no model will exactly
describe the natural phenomena we seek to analyze; in other words, ``all models
are wrong''. Thus we would like to uncover where and how the model goes wrong.

Edward explores model and inference criticism using
\begin{itemize}
  \item point-based evaluations, such as mean squared error or
  classification accuracy
\end{itemize}
\begin{lstlisting}[language=Python]
ed.evaluate('mean_squared_error', data={y: y_data, x: x_data})
\end{lstlisting}
\begin{itemize}
  \item posterior predictive checks, for making probabilistic
  assessments of the model fit using discrepancy functions
\end{itemize}
\begin{lstlisting}[language=Python]
T = lambda xs, zs: tf.reduce_mean(xs[x])
ed.ppc(T, data={x: x_data})
\end{lstlisting}

Criticism techniques are simply functions which take as input data,
the probability model and variational model (binded through a latent
variable dictionary), and any additional inputs.

\begin{lstlisting}[language=Python]
def criticize(data, latent_vars, ...)
  ...
\end{lstlisting}

Developing new criticism techniques is easy.  They can be derived from
the current techniques or built as a standalone function.

For examples of criticism techniques built in Edward, see the
criticism
\href{/tutorials/}{tutorials}.

{{autogenerated}}
