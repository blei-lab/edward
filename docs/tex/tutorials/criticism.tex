\title{Model Criticism}

\subsection{Model Criticism}

We can never validate whether a model is true. In practice, ``all
models are wrong'' \citep{box1976science}. However, we can try to
uncover where the model goes wrong. Model criticism helps justify the
model as an approximation or point to good directions for revising the
model.

Model criticism typically analyzes the posterior predictive
distribution,
\begin{align*}
  p(\mathbf{x}_\text{new} \mid \mathbf{x})
  &=
  \int
  p(\mathbf{x}_\text{new} \mid \mathbf{z})
  p(\mathbf{z} \mid \mathbf{x})
  \text{d} \mathbf{z}.
\end{align*}
The model's posterior predictive can be used to generate new data
given past observations and can also make predictions on new data
given past observations.
It is formed by calculating the likelihood of the new data, averaged
over every set of latent variables according to the posterior
distribution.

A helpful utility function to form the posterior predictive is
\texttt{copy}. For example, assume the model defines a likelihood
\texttt{x} connected to a prior
\texttt{z}. The posterior predictive distribution is
\begin{lstlisting}[language=Python]
x_post = ed.copy(x, {z: qz})
\end{lstlisting}
Here, we copy the likelihood node \texttt{x} in the graph and replace dependence
on the prior \texttt{z} with dependence on the inferred posterior \texttt{qz}.
We describe several techniques for model criticism.

\subsubsection{Point Evaluation}

A point evaluation is a scalar-valued metric for assessing
trained models \citep{winkler1996scoring,gneiting2007strictly}.
For example, we can assess models for classification
by predicting the label for each observation in the data and comparing
it to their true labels. Edward implements a variety of metrics, such
as classification error and mean absolute error.

The \texttt{ed.evaluate()} method takes as input a set of metrics to
evaluate, and a data dictionary. As with inference, the data dictionary binds the
observed random variables in the model to realizations: in this case,
it is the posterior predictive random variable of outputs \texttt{y_post} to
\texttt{y_train} and a placeholder for inputs \texttt{x} to
\texttt{x_train}.
\begin{lstlisting}[language=Python]
ed.evaluate('categorical_accuracy', data={y_post: y_train, x: x_train})
ed.evaluate('mean_absolute_error', data={y_post: y_train, x: x_train})
\end{lstlisting}

Point evaluation also applies to unsupervised tasks. For example, we
can evaluate the likelihood of observing the data.
\begin{lstlisting}[language=Python]
ed.evaluate('log_likelihood', data={x_post: x_train})
\end{lstlisting}

It is common practice to criticize models with data held-out from
training. To do this, we must first perform inference over any local
latent variables of the held-out data, fixing the global variables; we
demonstrate this below. Then we make predictions on the held-out data.

\begin{lstlisting}[language=Python]
from edward.models import Categorical

# create local posterior factors for test data, assuming test data
# has N_test many data points
qz_test = Categorical(logits=tf.Variable(tf.zeros[N_test, K]))

# run local inference conditional on global factors
inference_test = ed.Inference({z: qz_test}, data={x: x_test, beta: qbeta})
inference_test.run()

# build posterior predictive on test data
x_post = ed.copy(x, {z: qz_test, beta: qbeta}})
ed.evaluate('log_likelihood', data={x_post: x_test})
\end{lstlisting}

Point evaluations are formally known as scoring rules
in decision theory. Scoring rules are useful for model comparison, model
selection, and model averaging.

See the \href{/api/criticism}{criticism API} for further details.
An example of point evaluation is in the
\href{/tutorials/supervised-regression}{supervised learning
(regression)} tutorial.

\subsubsection{Posterior predictive checks}

Posterior predictive checks (PPCs)
analyze the degree to which data generated from the model deviate from
data generated from the true distribution. They can be used either
numerically to quantify this degree, or graphically to visualize this
degree. PPCs can be thought of as a probabilistic generalization of
point evaluation
\citep{box1980sampling,rubin1984bayesianly,meng1994posterior,gelman1996posterior}.

The simplest PPC works by applying a test statistic on new data
generated from the posterior predictive, such as
$T(\mathbf{x}_\text{new}) = \max(\mathbf{x}_\text{new})$.  Applying
$T(\mathbf{x}_\text{new})$ to new data over many data replications
induces a distribution. We compare this distribution to the test
statistic applied to the real data $T(\mathbf{x})$.

\includegraphics{/images/ppc.png}

In the figure, $T(\mathbf{x})$ falls in a low probability region of
this reference distribution. This indicates that the model fits the
data poorly according to this check; this suggests an area of
improvement for the model.

More generally, the test statistic can also be a function of the
model's latent variables $T(\mathbf{x}, \mathbf{z})$, known as a
discrepancy function.  Examples of discrepancy functions are the
metrics used for point evaluation. We can now interpret the
point evaluation as a special case of PPCs: it simply calculates
$T(\mathbf{x}, \mathbf{z})$ over the real data and without a reference
distribution in mind. A reference distribution allows us to make
probabilistic statements about the point, in reference to an overall
distribution.

The \texttt{ed.ppc()} method provides a scaffold for studying
various discrepancy functions.
\begin{lstlisting}[language=Python]
ed.ppc(lambda xs, zs: tf.reduce_mean(xs[x_post]), data={x_post: x_train})
\end{lstlisting}
The discrepancy can also take latent variables as input, which we pass
into the PPC.
\begin{lstlisting}[language=Python]
ed.ppc(lambda xs, zs: tf.maximum(zs[z]),
       data={y_post: y_train, x_ph: x_train},
       latent_vars={z: qz, beta: qbeta})
\end{lstlisting}

See the \href{/api/criticism}{criticism API} for further details.

PPCs are an excellent tool for revising models---simplifying or
expanding the current model as one examines its fit to data.
They are inspired by classical hypothesis testing; these methods
criticize models under the frequentist perspective of large sample
assessment.

PPCs can also be applied to tasks such as hypothesis testing, model
comparison, model selection, and model averaging.  It's important to
note that while PPCs can be applied as a form of Bayesian hypothesis
testing, hypothesis testing is generally not recommended: binary
decision making from a single test is not as common a use case as one
might believe. We recommend performing many PPCs to get a holistic
understanding of the model fit.

\subsubsection{References}\label{references}
