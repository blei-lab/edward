\title{Batch Training}

\subsection{Batch Training}

Running algorithms which require the full data set for each update
can be expensive when the data is large. In order to scale inferences,
we can do \emph{batch training}. This trains the model using
only a subsample of data at a time.

In this tutorial, we extend the
\href{http://edwardlib.org/tutorials/supervised-regression}
{supervised learning tutorial},
where the task is to infer hidden structure from
labeled examples $\{(x_n, y_n)\}$.
An interactive version with Jupyter notebook is available
\href{http://nbviewer.jupyter.org/github/blei-lab/edward/blob/master/notebooks/batch_training.ipynb}{here}.

\subsubsection{Data}

Simulate $N$ training examples and a fixed number of test examples.
Each example is a pair of inputs $\mathbf{x}_n\in\mathbb{R}^{10}$ and
outputs $y_n\in\mathbb{R}$. They have a linear dependence with
normally distributed noise.

\begin{lstlisting}[language=Python]
def build_toy_dataset(N, w):
  D = len(w)
  x = np.random.normal(0.0, 2.0, size=(N, D))
  y = np.dot(x, w) + np.random.normal(0.0, 0.05, size=N)
  return x, y

N = 10000  # size of training data
M = 128    # batch size during training
D = 10     # number of features

w_true = np.ones(D) * 5
X_train, y_train = build_toy_dataset(N, w_true)
X_test, y_test = build_toy_dataset(235, w_true)
\end{lstlisting}

We also define a helper function to select the next batch of data
points from the full set of examples. It keeps track of the current
batch index and returns the next batch using the function
\texttt{next()}.

\begin{lstlisting}[language=Python]
def generator(arrays, batch_size):
  """Generate batches, one with respect to each array's first axis."""
  starts = [0] * len(arrays)  # pointers to where we are in iteration
  while True:
    batches = []
    for i, array in enumerate(arrays):
      start = starts[i]
      stop = start + batch_size
      diff = stop - array.shape[0]
      if diff <= 0:
        batch = array[start:stop]
        starts[i] += batch_size
      else:
        batch = np.concatenate((array[start:], array[:diff]))
        starts[i] = diff
      batches.append(batch)
    yield batches

data = generator([X_train, y_train], M)
\end{lstlisting}

We will generate batches from \texttt{data} during inference.

\subsubsection{Model}

Posit the model as Bayesian linear regression \citep{murphy2012machine}.
For a set of $N$ data points $(\mathbf{X},\mathbf{y})=\{(\mathbf{x}_n, y_n)\}$,
the model posits the following distributions:

\begin{align*}
  p(\mathbf{w})
  &=
  \text{Normal}(\mathbf{w} \mid \mathbf{0}, \sigma_w^2\mathbf{I}),
  \\[1.5ex]
  p(b)
  &=
  \text{Normal}(b \mid 0, \sigma_b^2),
  \\
  p(\mathbf{y} \mid \mathbf{w}, b, \mathbf{X})
  &=
  \prod_{n=1}^N
  \text{Normal}(y_n \mid \mathbf{x}_n^\top\mathbf{w} + b, \sigma_y^2).
\end{align*}

The latent variables are the linear model's weights $\mathbf{w}$ and
intercept $b$, also known as the bias.
Assume $\sigma_w^2,\sigma_b^2$ are known prior variances and $\sigma_y^2$ is a
known likelihood variance. The mean of the likelihood is given by a
linear transformation of the inputs $\mathbf{x}_n$.

Let's build the model in Edward, fixing $\sigma_w,\sigma_b,\sigma_y=1$.

\begin{lstlisting}[language=Python]
X = tf.placeholder(tf.float32, [None, D])
y_ph = tf.placeholder(tf.float32, [None])

w = Normal(loc=tf.zeros(D), scale=tf.ones(D))
b = Normal(loc=tf.zeros(1), scale=tf.ones(1))
y = Normal(loc=ed.dot(X, w) + b, scale=1.0)
\end{lstlisting}

Here, we define a placeholder \texttt{X}. During inference, we pass in
the value for this placeholder according to batches of data.
To enable training with batches of varying size,
we don't fix the number of rows for \texttt{X} and \texttt{y}. (Alternatively,
we could fix it to be the batch size if we're training and testing
with a fixed size.)

\subsubsection{Inference}

We now turn to inferring the posterior using variational inference.
Define the variational model to be a fully factorized normal across
the weights.
\begin{lstlisting}[language=Python]
qw = Normal(loc=tf.Variable(tf.random_normal([D])),
            scale=tf.nn.softplus(tf.Variable(tf.random_normal([D]))))
qb = Normal(loc=tf.Variable(tf.random_normal([1])),
            scale=tf.nn.softplus(tf.Variable(tf.random_normal([1]))))
\end{lstlisting}

Run variational inference with the Kullback-Leibler divergence.
We use $5$ latent variable samples for computing
black box stochastic gradients in the algorithm.
(For more details, see the
\href{/tutorials/klqp}{$\text{KL}(q\|p)$ tutorial}.)

For batch training, we iterate over the number of batches and
feed them to the respective placeholder. We set the number of
iterations to be the total number of batches for 5 epochs
(full passes over the data set).

\begin{lstlisting}[language=Python]
n_batch = int(N / M)
n_epoch = 5

inference = ed.KLqp({w: qw, b: qb}, data={y: y_ph})
inference.initialize(
    n_iter=n_batch * n_epoch, n_samples=5, scale={y: N / M})
tf.global_variables_initializer().run()

for _ in range(inference.n_iter):
  X_batch, y_batch = next(data)
  info_dict = inference.update({X: X_batch, y_ph: y_batch})
  inference.print_progress(info_dict)
\end{lstlisting}

\begin{lstlisting}
390/390 [100%] ██████████████████████████████ Elapsed: 4s | Loss: 10481.556
\end{lstlisting}

When initializing inference, note we scale $y$ by $N/M$, so it is as if the
algorithm had seen $N/M$ as many data points per iteration.
Algorithmically, this will scale all computation regarding $y$ by
$N/M$ such as scaling the log-likelihood in a variational method's
objective. (Statistically, this avoids inference being dominated by the prior.)

The loop construction makes training very flexible. For example, we
can also try running many updates for each batch.

\begin{lstlisting}[language=Python]
n_batch = int(N / M)
n_epoch = 1

inference = ed.KLqp({w: qw, b: qb}, data={y: y_ph})
inference.initialize(n_iter=n_batch * n_epoch * 10, n_samples=5, scale={y: N / M})
tf.global_variables_initializer().run()

for _ in range(inference.n_iter // 10):
  X_batch, y_batch = next(data)
  for _ in range(10):
    info_dict = inference.update({X: X_batch, y_ph: y_batch})

  inference.print_progress(info_dict)
\end{lstlisting}

\begin{lstlisting}
770/780 [ 98%] █████████████████████████████  ETA: 0s | Loss: 9760.541
\end{lstlisting}

In general, make sure that the total number of training iterations is
specified correctly when initializing \texttt{inference}. Otherwise an incorrect
number of training iterations can have unintended consequences; for example,
\texttt{ed.KLqp} uses an internal counter to appropriately decay its optimizer's
learning rate step size.

Note also that the reported \texttt{loss} value as we run the
algorithm corresponds to the computed objective given the current
batch and not the total data set. We can instead have it report
the loss over the total data set by summing \texttt{info_dict['loss']}
for each epoch.

\subsubsection{Criticism}

A standard evaluation for regression is to compare prediction accuracy on
held-out ``testing'' data. We do this by first forming the posterior predictive
distribution.
\begin{lstlisting}[language=Python]
y_post = ed.copy(y, {w: qw, b: qb})
# This is equivalent to
# y_post = Normal(loc=ed.dot(X, qw) + qb, scale=tf.ones(N))
\end{lstlisting}

With this we can evaluate various quantities using predictions from
the model (posterior predictive).
\begin{lstlisting}[language=Python]
print("Mean squared error on test data:")
print(ed.evaluate('mean_squared_error', data={X: X_test, y_post: y_test}))

print("Mean absolute error on test data:")
print(ed.evaluate('mean_absolute_error', data={X: X_test, y_post: y_test}))
\end{lstlisting}

\begin{lstlisting}
## Mean squared error on test data:
## 0.00659598
## Mean absolute error on test data:
## 0.0705906
\end{lstlisting}

The trained model makes predictions with low error
(relative to the magnitude of the output).

\subsubsection{Footnotes}

Only certain algorithms support batch training such as
\texttt{MAP}, \texttt{KLqp}, and \texttt{SGLD}. Also, above we
illustrated batch training for models with only global latent variables,
which are variables are shared across all data points.
For more complex strategies, see the
\href{http://edwardlib.org/api/inference-data-subsampling} {inference
data subsampling API}.

\subsubsection{References}\label{references}
