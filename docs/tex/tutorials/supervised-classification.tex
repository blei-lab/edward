\title{Supervised Learning (Classification)}

\subsection{Supervised Learning (Classification)}

In supervised learning, the task is to infer hidden structure from
labeled data, comprised of training examples $\{(x_n, y_n)\}$.
Classification means the output $y$ takes discrete values.

We demonstrate with an example in Edward.
An interactive version with Jupyter notebook is available
\href{http://nbviewer.jupyter.org/github/blei-lab/edward/blob/master/docs/notebooks/supervised_classification.ipynb}{here},
or as a script available at
\href{https://github.com/blei-lab/edward/blob/master/examples/gp_classification.py}
{\texttt{examples/gp_classification.py}} in the Github repository.

\subsubsection{Data}

Use 25 data points from the \href
{https://stat.ethz.ch/R-manual/R-devel/library/MASS/html/crabs.html}
{crabs data set}.
\begin{lstlisting}[language=Python]
df = np.loadtxt('data/crabs_train.txt', dtype='float32', delimiter=',')
df[df[:, 0] == -1, 0] = 0  # replace -1 label with 0 label

N = 25  # number of data points
D = df.shape[1] - 1  # number of features

subset = np.random.choice(df.shape[0], N, replace=False)
X_train = df[subset, 1:]
y_train = df[subset, 0]
\end{lstlisting}

\subsubsection{Model}

A Gaussian process is a powerful object for modeling nonlinear
relationships between pairs of random variables. It defines a distribution over
(possibly nonlinear) functions, which can be applied for representing
our uncertainty around the true functional relationship.
Here we define a Gaussian process model for classification
\citep{rasmussen2006gaussian}.

Formally, a distribution over functions $f:\mathbb{R}^D\to\mathbb{R}$ can be specified
by a Gaussian process
\begin{align*}
  p(f)
  &=
  \mathcal{GP}(f\mid \mathbf{0}, k(\mathbf{x}, \mathbf{x}^\prime)),
\end{align*}
whose mean function is the zero function, and whose covariance
function is some kernel which describes dependence between
any set of inputs to the function.

Given a set of input-output pairs
$\{\mathbf{x}_n\in\mathbb{R}^D,y_n\in\mathbb{R}\}$,
the likelihood can be written as a multivariate normal
\begin{align*}
  p(\mathbf{y})
  &=
  \text{Normal}(\mathbf{y} \mid \mathbf{0}, \mathbf{K})
\end{align*}
where $\mathbf{K}$ is a covariance matrix given by evaluating
$k(\mathbf{x}_n, \mathbf{x}_m)$ for each pair of inputs in the data
set.

The above applies directly for regression where $\mathbb{y}$ is a
real-valued response, but not for (binary) classification, where $\mathbb{y}$
is a label in $\{0,1\}$. To deal with classification, we interpret the
response as latent variables which is squashed into $[0,1]$. We then
draw from a Bernoulli to determine the label, with probability given
by the squashed value.

Define the likelihood of an observation $(\mathbf{x}_n, y_n)$ as
\begin{align*}
  p(y_n \mid \mathbf{z}, x_n)
  &=
  \text{Bernoulli}(y_n \mid \text{logit}^{-1}(\mathbf{x}_n^\top \mathbf{z})).
\end{align*}

Define the prior to be a multivariate normal
\begin{align*}
  p(\mathbf{z})
  &=
  \text{Normal}(\mathbf{z} \mid \mathbf{0}, \mathbf{K}),
\end{align*}
with covariance matrix given as previously stated.

Let's build the model in Edward. We use a radial basis function (RBF)
kernel, also known as the squared exponential or exponentiated
quadratic.
\begin{lstlisting}[language=Python]
from edward.models import Bernoulli, Normal
from edward.util import multivariate_rbf

def kernel(x):
  mat = []
  for i in range(N):
    mat += [[]]
    xi = x[i, :]
    for j in range(N):
      if j == i:
        mat[i] += [multivariate_rbf(xi, xi)]
      else:
        xj = x[j, :]
        mat[i] += [multivariate_rbf(xi, xj)]

    mat[i] = tf.stack(mat[i])

  return tf.stack(mat)

X = tf.placeholder(tf.float32, [N, D])
f = MultivariateNormalFull(mu=tf.zeros(N), sigma=kernel(X))
y = Bernoulli(logits=f)
\end{lstlisting}
Here, we define a placeholder \texttt{X}. During inference, we pass in
the value for this placeholder according to data.

\subsubsection{Inference}

Perform variational inference.
Define the variational model to be a fully factorized normal.
\begin{lstlisting}[language=Python]
qf = Normal(mu=tf.Variable(tf.random_normal([N])),
            sigma=tf.nn.softplus(tf.Variable(tf.random_normal([N]))))
\end{lstlisting}

Run variational inference for \texttt{500} iterations.
\begin{lstlisting}[language=Python]
inference = ed.KLqp({f: qf}, data={X: X_train, y: y_train})
inference.run(n_iter=500)
\end{lstlisting}
In this case
\texttt{KLqp} defaults to minimizing the
$\text{KL}(q\|p)$ divergence measure using the reparameterization
gradient.
For more details on inference, see the \href{/tutorials/klqp}{$\text{KL}(q\|p)$ tutorial}.
(This example happens to be slow because evaluating and inverting full
covariances in Gaussian processes happens to be slow.)

% \subsubsection{Criticism}

\subsubsection{References}\label{references}
